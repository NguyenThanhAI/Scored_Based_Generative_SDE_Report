\documentclass{article} % For LaTeX2e
\usepackage{iclr2021_conference,times}

% Optional math commands from https://github.com/goodfeli/dlbook_notation.
\input{math_commands.tex}

\usepackage{amsmath}
\usepackage{amssymb}
\usepackage{amsthm}
\usepackage{thmtools}
\usepackage{mathtools}
\usepackage{graphicx}
%\usepackage{algpseudocode}
%\usepackage{algorithm}
\usepackage[ruled,vlined,linesnumbered,algosection]{algorithm2e}
\usepackage{blindtext}
\usepackage{setspace}
\usepackage[utf8]{inputenc}
\usepackage[utf8]{vietnam}
\usepackage[center]{caption}
\usepackage[shortlabels]{enumitem}
\usepackage{fancyhdr} % header, footer
\usepackage{hyperref} % loại bỏ border với mục lục và công thức
\usepackage{url}
\usepackage[nonumberlist, nopostdot, nogroupskip]{glossaries}
\usepackage{glossary-superragged}
\usepackage{tikz,tkz-tab}
%\usepackage[style=numeric,sortcites]{biblatex}
%\addbibresource{ref.bib}
%\usepackage[numbers]{natbib}
\usepackage{indentfirst}
\usepackage{multirow}
\usepackage{cancel}

\graphicspath{{./figures/}}

\title{Mô hình sinh dựa trên điểm số bằng \\ phương trình vi phân ngẫu nhiên}


\author{Nguyễn Chí Thanh \\
Khoa Toán - Cơ - Tin học\\
Trường Đại học Khoa học Tự nhiên\\
Đại học Quốc Gia Hà Nội \\
\texttt{nguyenchithanh\_sdh21@hus.edu.vn} \\
}

\newcommand{\fix}{\marginpar{FIX}}
\newcommand{\new}{\marginpar{NEW}}

\iclrfinalcopy % Uncomment for camera-ready version, but NOT for submission.

\begin{document}
\maketitle

\begin{abstract}
    Tạo nhiễu từ dữ liệu là một việc đơn giản, nhưng tạo dữ liệu từ nhiễu được gọi là mô hình sinh.
    Ở đề tài này, ta sẽ trình bày phương trình vi phân ngẫu nhiên (SDE) biến đổi một phân phối dữ liệu phức tạp thành một phân phối tiên nghiệm biết trước trơn và thông suốt bằng cách dần dần thêm nhiễu vào dữ liệu,
    và một phương trình vi phân ngẫu nhiên (SDE) ngược (theo thời gian) biến đổi từ phân phối tiên nghiệm trở lại phân phối dữ liệu một cách từ từ bằng việc loại bỏ dần nhiễu.
    Một điều quan trọng là phương trình vi phân ngược theo thời gian chỉ phụ thuộc vào trường gradient phụ thuộc vào thời gian (được gọi là điểm số) của phân phối dữ liệu bị xáo trộn bởi nhiễu.
    Bằng việc tận dụng những bước tiến trong mô hình sinh dựa trên điểm, ta có thể ước lượng chính xác điểm với mạng neuron, và sử dụng các bộ giải phương trình vi phân ngẫu nhiên (SDE) bằng phương pháp số để sinh ra các mẫu dữ liệu.
    Ta sẽ chỉ ra rằng khung làm việc này bao gồm các cách tiếp cận trước đây về mô hình sinh dựa trên điểm số và các mô hình khuếch tán xác suất, cho phép các thủ tục mới để sinh dữ liệu ra đời với những khả năng mạnh hơn.
    Đặc biệt, ta sẽ giới thiệu một bộ dự đoán - căn chỉnh để căn chỉnh sai lệch trong quá trình biến đổi của phương trình vi phân ngẫu nhiên ngược thời gian được rời rạc hóa.
    Ta cũng thu được một mạng neuron ODE tương đương được lấy mẫu từ cùng phân phối như SDE, nhưng ngoài ra cũng cho phép tính toán chính xác độ hợp lý, và cải thiện độ hiệu quả của quá trình lấy mẫu.
    Ngoài ra, ta cũng cung cấp một cách mới để giải các bài toán ngược với các mô hình dựa trên điểm số, được giải thích với các thí nghiệm trên sinh mẫu dựa trên điều kiện nhãn, vẽ ảnh và tô màu.
    Kết hợp với nhiều bước tiến về các kiến trúc mô hình, ta đã đạt kỷ lục với việc sinh ảnh trên tập CIFAR-10 với điểm Inception score là 9.89 và FID là 2.20 và độ hợp lý rất tốt 2.99 bits/dim, và thể hiện khả năng sinh ảnh chân thực với ảnh có độ phân giải 1024 $\times$ 1024 lần đầu tiên từ một mô hình sinh dựa trên điểm số.
\end{abstract}

\section{GIỚI THIỆU}

Hai lớp thành công của các mô hình sinh xác suất liên quan đến việc làm biến đổi dữ liệu một cách tuần tự với độ nhiễu tăng dần, sau đó học cách đảo ngược sự biến đổi để tạ thành một mô hình sinh của dữ liệu.
\textit{Score matching with Langevin dynamics} (SMLD) \citep{song2019generative} ước lượng điểm số (ví dụ gradient của log của hàm mật độ xác suất của dữ liệu) tại từng mức nhiễu, sau đó sử dụng động học Langevin để lấy mẫu từ mỗi dãy các mức nhiễu giảm dần trong suốt quá trình sinh dữ liệu.
\textit{Denoising diffusion probabilistic modeling} (DDPM) \citep{sohl2015deep,ho2020denoising} huấn luyện một dãy các mô hình xác suất để đảo ngược từng bước của quá trình bị xáo trộn bởi nhiễu,
sử dụng những kiến thức về dạng chức năng của các phân phối ngược làm cho quá trình huấn luyện dễ dàng hơn.
Trong không gian trạng thái liên tục, DDPM huấn luyện hàm mục tiêu ngầm tính điểm số tại từng mức nhiễu.
Vì vậy, ta gọi hai lớp mô hình này là các \textit{mô hình sinh dựa trên điểm số}.

Các mô hình sinh dựa trên điểm số và các kỹ thuật liên quan \citep{bordes2017learning, goyal2017variational,du2019implicit} đã được chứng minh là hiệu quả trong bài toán sinh ảnh \citep{song2019generative,song2020sliced,ho2020denoising}, âm thanh \citep{chen2020wavegrad,kong2020diffwave}, đồ thị \citep{niu2020permutation} và các khối hình \citep{cai2020learning}.
Để cho phép các phương pháp lấy mẫu mới mở rộng khả năng của các mô hình sinh dựa trên điểm số, ta đề xuất một khung làm việc thống nhất tổng quát hóa các cách tiếp cận trước đó thông qua góc nhìn của phương trình vi phân ngẫu nhiên.

\begin{figure}[h!]
    \centering
    \includegraphics[width=0.8\textwidth]{1.jpg}
    \caption{\textbf{Giải một phương trình SDE đảo ngược thời gian thu được một mô hình sinh dựa trên điểm.}
    Biến đổi dữ liệu thành một phân phối nhiễu đơn giản có thể được thực hiện với một phương trình SDE liên tục theo thời gian.
    Phương trình SDE này có thể được đảo ngược nếu ta biết điểm của phân phối tại từng bước thời gian trung gian, $\nabla_{\bold{x}}\log p_t (\bold{x})$.}
    \label{fig:1}
\end{figure}

Cụ thể, thay vì làm nhiễu với một số hữu hạn của các phân phối nhiễu, ta xét một phân phối biến đổi một cách liên tục theo thời gian tương ứng với quá trình khuếch tán.
Quá trình này khuếch tán dần dần các điểm dữ liệu thành nhiễu ngẫu nhiên và được cho bởi một SDE cho trước không phụ thuộc vào dữ liệu và không có các tham số có thể huấn luyện.
Bằng cách đảo ngược quá trình này, ta có thể biến đổi nhiễu thành dữ liệu một cách trơn cho bài toán sinh mẫu dữ liệu.
Điều quan trọng là quá trình ngược này thỏa mãn SDE ngược theo thời gian \citep{anderson1982reverse}, có thể suy ra từ phương trình SDE thuận được cho bởi điểm số của hàm mật độ xác suất cận biên như là một hàm của thời gian.
Vì vậy ta có thể xấp xỉ SDE ngược thời gian bằng cách huấn luyện một mạng neuron phụ thuộc thời gian để ước lượng điểm số và sau đó tạo ra các mẫu dữ liệu sử dụng các bộ giải SDE bằng phương pháp số.
Ý tưởng chính được trình bày ở hình \ref{fig:1}.

Khung làm việc đề xuất có nhiều đóng góp về mặt lý thuyết và thực tiễn:

\textbf{Lấy mẫu linh hoạt và tính toán độ hợp lý:} Ta có thể sử dụng bất kỳ một bộ giải phương trình SDE đa chức năng nào để tích hợp phương trình SDE đảo ngược thời gian cho bài toán lấy mẫu.
Hơn nữa, ta đề xuất hai phương pháp đặc biệt mà không phù hợp với phương trình SDE nói chung: (i) Bộ lấy mẫu Dự đoán - Căn chỉnh (PC) kết hợp với bộ giải SDE bằng phương pháp số và cách tiếp cận MCMC (Markov Chain Monte Carlo) dựa trên điểm số, như Langevin MCMC \citep{parisi1981correlation} và HMC \citep{neal2011mcmc} và (ii) các bộ lấy mẫu tất định dựa trên dòng xác suất của các phương trình vi phân thường (ODE).
Phương pháp thứ nhất hợp nhất vả cải thiện các phương pháp lấy mẫu hiện có cho các mô hình dựa trên điểm số.
Phương pháp thứ hai cho phép lấy mẫu thích ứng nhanh thông qua một hộp đen là một bộ giải phương trình ODE, các thao tác trên dữ liệu linh hoạt thông qua mã ẩn, mỗi mã có tính duy nhất và điều đáng chú ý ta có thể tính chính xác độ hợp lý.

\textbf{Sinh dữ liệu có điều kiện:} Ta có thể điều chỉnh quá trình sinh dữ liệu bằng cách điều chỉnh các thông tin không có sẵn trong quá trình huấn luyện, bởi vì phương trình SDE đảo ngược thời gian có điều kiện có thể được ước lượng hiệu quả từ các điểm số \textit{không có điều kiện}.
Điều này cho phép các ứng dụng như sinh ảnh có điều kiện về lớp, vẽ ảnh, tô màu cũng như các bài toán ngược khác, tất cả các bài toán trên đều có thể được giải quyết bằng cách sử dụng một mô hình dựa trên điểm số không có điều kiện mà không cần phải huấn luyện lại mô hình.

\textbf{Khung làm việc thống nhất:} Khung làm việc của ta cung cấp một cách thống nhất để khai phá của chỉnh định nhiều phương trình SDE cho việc cải thiện các mô hình sinh dựa trên điểm số.
Các phương pháp của SMLD và DDPM có thể được đưa vào khung làm việc của ta bằng cách rời rạc hóa hai phương trình SDE riêng biệt.
Mặc dù DDPM \citep{ho2020denoising} gần đây đã được báo cáo là có thể đạt được mẫu có chất lượng cao hơn SMLD \citep{song2019generative,song2020improved}, ta chỉ ra điều đó với kiến trúc mạng tốt hơn và các thuật toán lấy mẫu mới cho phép bởi khung làm việc của ta, mô hình sau đó có thể bắt kịp và đạt được đến điểm Inception kỷ lục (9.89) và FID (2.20) trên tập CIFAR-10 cũng như độ chân thực của ảnh có độ phân giải 1024 $\times$ 1024 lần đầu tiên từ một mô hình sinh dựa trên điểm.
Ngoài ra, ta đề xuất một phương trình SDE mới trong khung làm việc của ta thu được độ hợp lý có giá trị 2.99 bits/dim trên tập CIFAR-10, tạo nên một kỷ lục mới cho bài toán này.

\section{Kiến thức nền tảng}

\subsection{Khử nhiều khớp điểm số với động học Langevin (SMLD)}

Đặt $p_{\sigma}(\tilde{\bold{x}} \vert \bold{x}) \vcentcolon = \mathcal{N}(\tilde{\bold{x}};\bold{x}, \sigma^2 \bold{I})$ là một nhân tạo ra nhiễu, và $p_{\sigma}(\tilde{\bold{x}})\vcentcolon = \int p_{\mathrm{data}}(\bold{x}) p_{\sigma}(\tilde{\bold{x}} \vert \bold{x}) d \bold{x}$, với $p_{\mathrm{data}}$ ký hiệu là phân phối dữ liệu.
Ta xét một dãy các mức nhiễu dương $\sigma_{\min} < \sigma_2 < \dots < \sigma_N = \sigma_{\max}$.
Thông thường, $\sigma_{\min}$ đủ nhỏ dể $p_{\sigma_{\min}} \approx p_{\mathrm{data}}(\bold{x})$, và $\sigma_{\max}$ đủ lớn để $p_{\sigma_{\max}}\approx \mathcal{N}(\bold{x}; \bold{0}, \sigma_{\max}^2 \bold{I})$.
\citep{song2019generative} đề xuất huấn luyện một mạng điểm số có điều kiện nhiễu (Noise Conditional Score Network - NCSN), ký hiệu $\bold{s}_{\boldsymbol{\theta}}(\bold{x}, \sigma)$ với một tổng có trọng số của các hàm khớp điểm số khử nhiễu \citep{vincent2011connection}:

\begin{equation}
    \boldsymbol{\theta}^{\ast} = \argmin_{\boldsymbol{\theta}} \sum_{i=1}^N \sigma_i^2 \mathbb{E}_{p_{\mathrm{data}}(\bold{x})} \mathbb{E}_{p_{\sigma_i}(\tilde{\bold{x}} \vert \bold{x})} \big\lbrack \lVert \bold{s}_{\boldsymbol{\theta} (\tilde{\bold{x}}, \sigma_i)} - \nabla_{\tilde{\bold{x}}} \log p_{\sigma_i} (\tilde{\bold{x}} \vert \bold{x})  \rVert_2^2 \big\rbrack
\end{equation}

Khi được cho một dữ liệu và một mô hình có kích thước phù hợp, mô hình dựa trên điểm số tối ưu $\bold{s}_{\boldsymbol{\theta}^{\ast}}(\bold{x}, \sigma)$ khớp với $\nabla_{\bold{x}} \log p_{\sigma}(\bold{x})$ tại hầu hết vị trí với $\sigma \in \lbrace \sigma_i \rbrace_{i=1}^N$.
Để lấy mẫu, \citep{song2019generative} chạy M bước của Langevin MCMC để thu được mẫu dữ liệu cho lần lượt từng $p_{\sigma_i}(\bold{x})$:

\begin{equation}
    \bold{x}_i^m = \bold{x}_i^{m-1} + \epsilon_i \bold{s}_{\boldsymbol{\theta}^{\ast}} (\bold{x}_i^{m-1}, \sigma_i) + \sqrt{2 \epsilon_i} \bold{z}_i^m, m = 1, 2, \dots, M
\end{equation}

với $\epsilon_i > 0$ là độ dài bước và $\bold{z}_i^m$ là một biến tuân theo phân phối chuẩn tắc.
Công thức trên được lặp lại cho $i=N, N-1, \dots, 1$ với $\bold{x}_N^0 \sim \mathcal{N}(\bold{x} \vert \bold{0}, \sigma_{\max}^2 \bold{I})$ và $\bold{x}_i^0 = \bold{x}_{i+1}^M$ khi $i < N$. Khi $M \rightarrow \infty$ và $\epsilon_i \rightarrow 0$ với mọi $i, \bold{x}_1^M$ trở thành một mẫu chính xác từ $p_{\sigma_{\min}}(\bold{x})\approx p_{\mathrm{data}}(\bold{x})$ trong một số điều kiện.


\subsection{Mô hình xác suất khuếch tán khử nhiễu (DDPM)}

\citep{sohl2015deep,ho2020denoising} xét một dãy các ngưỡng nhiễu dương $0 < \beta_1, \beta_2, \dots, \beta_N < 1$.
Với từng điểm dữ liệu huấn luyện $\bold{x}_0 \sim p_{\mathrm{data}}(\bold{x})$, một xích Markov rời rạc $\lbrace \bold{x}_1, \bold{x}_2, \dots, \bold{x}_N \rbrace$ được tạo ra sao cho $p(\bold{x}_i \vert \bold{x}_{i-1}) = \mathcal{N}(\bold{x}_i; \sqrt{1-\beta}_i)$

\newpage
\bibliography{iclr2021_conference}
\bibliographystyle{iclr2021_conference}

\end{document}